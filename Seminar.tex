% --- Template for thesis / report with tktltiki2 class ---
% 
% last updated 2013/02/15 for tkltiki2 v1.02

\documentclass[english]{tktltiki2}

% tktltiki2 automatically loads babel, so you can simply
% give the language parameter (e.g. finnish, swedish, english, british) as
% a parameter for the class: \documentclass[finnish]{tktltiki2}.
% The information on title and abstract is generated automatically depending on
% the language, see below if you need to change any of these manually.
% 
% Class options:
% - grading                 -- Print labels for grading information on the front page.
% - disablelastpagecounter  -- Disables the automatic generation of page number information
%                              in the abstract. See also \numberofpagesinformation{} command below.
%
% The class also respects the following options of article class:
%   10pt, 11pt, 12pt, final, draft, oneside, twoside,
%   openright, openany, onecolumn, twocolumn, leqno, fleqn
%
% The default font size is 11pt. The paper size used is A4, other sizes are not supported.
%
% rubber: module pdftex

% --- General packages ---

\usepackage[utf8]{inputenc}
\usepackage[T1]{fontenc}
\usepackage{lmodern}
\usepackage{microtype}
\usepackage{amsfonts,amsmath,amssymb,amsthm,booktabs,color,enumitem,graphicx}
\usepackage[pdftex,hidelinks]{hyperref}
\usepackage{setspace}
\usepackage{subfig}
\usepackage{rotating,tabularx}
\usepackage{float}
\usepackage{rotating}
\usepackage{textgreek}
\usepackage{fixltx2e}
\usepackage{url}
% Automatically set the PDF metadata fields
\makeatletter
\AtBeginDocument{\hypersetup{pdftitle = {\@title}, pdfauthor = {\@author}}}
\makeatother

% --- Language-related settings ---
%
% these should be modified according to your language

% babelbib for non-english bibliography using bibtex
\usepackage[fixlanguage]{babelbib}
\selectbiblanguage{english}

% add bibliography to the table of contents
\usepackage[nottoc]{tocbibind}
% tocbibind renames the bibliography, use the following to change it back
\settocbibname{Sources}

% --- Theorem environment definitions ---

\newtheorem{lau}{Lause}
\newtheorem{lem}[lau]{Lemma}
\newtheorem{kor}[lau]{Korollaari}

\theoremstyle{definition}
\newtheorem{maar}[lau]{Määritelmä}
\newtheorem{ong}{Ongelma}
\newtheorem{alg}[lau]{Algoritmi}
\newtheorem{esim}[lau]{Esimerkki}

\theoremstyle{remark}
\newtheorem*{huom}{Huomautus}


% --- tktltiki2 options ---
%
% The following commands define the information used to generate title and
% abstract pages. The following entries should be always specified:

\title{Techniques for countering privacy threats of Location-based services}
\author{Lauri Suomalainen}
\date{\today}
\level{Seminar}
\abstract{Most of the social networking services today exploit location-based data in one way or another. Services like Facebook allow users to GeoTag their current location and tag themselves and their friends. In exchange the service can use the data to offer recommendations, news and so on.  While the users of the services are often aware that they are providing personal locational data, its pervasiveness and accuracy may come as a surprise and have serious repercussions when it comes to users' real-life privacy. A malicious actor such as a burglar could exploit the data a user provides to for example find out their street address and times when they are not home.
This seminar paper reviews and evaluates several techniques used to preserve and protect users' privacy when using Location-based services. The goal for the techniques is to counter several threats Location-based Services face in such manner that users can still use said services without compromising their security.}

% The following can be used to specify keywords and classification of the paper:

\keywords{Geo-location, Privacy, Social Networks}

% classification according to ACM Computing Classification System (http://www.acm.org/about/class/)
% This is probably mostly relevant for computer scientists
% uncomment the following; contents of \classification will be printed under the abstract with a title
% "ACM Computing Classification System (CCS):"
% \classification{}

% If the automatic page number counting is not working as desired in your case,
% uncomment the following to manually set the number of pages displayed in the abstract page:
%
% \numberofpagesinformation{16 sivua + 10 sivua liitteissä}
%
% If you are not a computer scientist, you will want to uncomment the following by hand and specify
% your department, faculty and subject by hand:
%
% \faculty{Matemaattis-luonnontieteellinen}
% \department{Tietojenkäsittelytieteen laitos}
% \subject{Tietojenkäsittelytiede}
%
% If you are not from the University of Helsinki, then you will most likely want to set these also:
%
% \university{Helsingin Yliopisto}
% \universitylong{HELSINGIN YLIOPISTO --- HELSINGFORS UNIVERSITET --- UNIVERSITY OF HELSINKI} % displayed on the top of the abstract page
% \city{Helsinki}
%


\begin{document}


% --- Front matter ---

\frontmatter      % roman page numbering for front matter

\maketitle        % title page
\makeabstract     % abstract page

\tableofcontents  % table of contents
\listoffigures
% --- Main matter ---

\mainmatter 
\onehalfspacing
      % clear page, start arabic page numbering

\section*{Abstract}
 \textit{Most of the social networking services today exploit location-based data in one way or another. Services like Facebook allow users to GeoTag their current location and tag themselves and their friends. In exchange the service can use the data to offer recommendations, news and so on.  While the users of the services are often aware that they are providing personal locational data, its pervasiveness and accuracy may come as a surprise and have serious repercussions when it comes to users' real-life privacy. A malicious actor such as a burglar could exploit the data a user provides to for example find out their street address and times when they are not home.
This seminar paper reviews and evaluates several techniques used to preserve and protect users' privacy when using Location-based services. The goal for the techniques is to counter several threats Location-based Services face in such manner that users can still use said services without compromising their security.}

\section{Introduction}

Location-Based Services (LBS) are applications that operate on geographical or other location based data. Original LBSs can be traced back to mid-20th century, but what is understood by LBS today is tied to the mid-21th century emergence of G3 networks and hand-held devices with GPS capability \cite{History}. Many contemporary LSBs take a form as a part of a social networking applications such as Facebook, Foursquare and Instagram. There are several ways LBSs work. They can make use of users' mobile device's GPS and provide services based on users' whereabouts when requested or continually monitor users' location \cite{LocationPrivacy}. Some services allow users to \textit{check-in} to pre-determined venues or points of interest (POI) whereas other allow more exact location with the GPS data. This practice is commonly known as \textit{GeoTagging}. \par
LBSs do not come without security risks. By publishing their location on the internet exposes users to threats in the real world. A malicious agent, for example a burglar, could monitor one user's LBS usage and e.g. determine when the user is not at their residence. Using that information the attacker could orchestrate a break-in without the risk of being caught red-handed \cite{Friedland2010}. While many applications require user to explicitly publish their locational data, users also often do it unknowingly. For example many contemporary smart phones as well as digital cameras automatically embed meta data to photographs and it can contain privacy sensitive information such as coordinates and the time the photograph was taken \cite{Friedland2010}. In some cases the user themselves does not expose their location data explicitly, but with certain LBSs supporting \textit{User Tagging}, a user's acquaintance can expose the user without them having a say for it. Thus a user can be associated with places, people and other information they would rather keep private. \par
The obvious privacy issues have been researched a lot and many techniques have been presented which address one or more possible privacy threats LBSs face \cite{LocationPrivacy}. These range from offering users options to rule how, where and when can their locational and identity data used to more technical solutions such as query enlargements and encryption-based techniques. Each of them have have their uses and shortcomings: They can only address a certain set of privacy threats and may come with pre-assumptions and computational costs. \par 
In this paper we first review threats LBSs face in section 2. Then in section 3 we take a look at the techniques designed to address different kinds of privacy threats in LBSs. Section 4 compares and evaluates the techniques in respect of their intended use as well as other techniques. Section 5 will be the concluding part summarising the state of privacy in Location-Based Services and also shortly addressing the future trends of the field. 

\section{Threats in Location-based Services}

A privacy threats in LBS can be defined as events during which an adversary can gain information about the user which they consider sensitive\cite{LocationPrivacy}. Threats fall into two major categories: Release of sensitive location information and re-identification through location information. In the first case, the identity of the user is known but the location information associated with that identity is considered sensitive and somehow made available for the adversary. In the latter case the user would like to have their identity kept secret but the adversary can exploit their location data to narrow down their possible identities in the set of other associated identities thus reducing their degree of anonymity. \cite{LocationPrivacy} list four different privacy threat vectors and \cite{Ghinita2008} add one more. The vectors are location privacy, absence privacy, co-location privacy, identity privacy and correlation privacy.

\subsection{Location Privacy}
Location privacy is the most straightforward of attack vectors. Whenever a user publishes their location they associate their identity with that location and possibly reveal information they consider sensitive. Even though a user can generalise their location it is not guaranteed to be safe. For example, a user would GeoTag themselves in a LBS whilst visiting a bar with a friend, but instead of using the exact location they would use the more general location like the part of town as the tag. Now if the aforementioned friend would happen to meet another friend at the bar too and GeoTag them both but now using the exact location of the bar, an adversary with the access of both GeoTags could easily deduct first user's exact location as well as the fact that they are accompanied by the third friend, both being pieces of information the first user did not explicitly wish to expose \cite{LocationPrivacy}.

\subsection{Absence Privacy}
Absence Privacy can be regarded as the other side of the location privacy. As the user publishes information about where they are at the given moment, they simultaneously expose explicitly where they are not. This threat factor also has a temporal dimension. For example, an adversary planning a break-in to a user's home can use the user's location information to estimate how long they will approximately be absent in order to avoid being caught while on the crime scene.

\subsection{Co-location Privacy}
Co-location Privacy refers to a privacy threat in which the adversary monitors multiple users' location information in order to associate other users with each other and different locations. For example, if user A GeoTags themselves and their friend user B to a location and simultaneously user C tags themselves and their friend user D at the same place, an adversary with the access to the information of only A and C can possibly violate the privacy of both B and D as being associated with someone on the other party at the given place can be considered sensitive by users.

\subsection{Identity Privacy}
Identity Privacy refers to a scenario in which a user employs an obfuscated identity such as a pseudonym to protect their real identity but their identity can be revealed by associating them with a certain location. For example, the user is a 30-year-old male who is using a location-based dating service at his workplace. In the service he uses a pseudonym because he considers the fact that he is using the service to be embarrassing and thus sensitive. Now if he happened to be the only male of 30 years at his workplace and his supervisor would happen to check out the dating service and only search for users in their immediate vicinity, the supervisor would find out that the 30-year-old user is using the service during work hours effectively violating the user's privacy. In a scenario like this the pseudonym does naught to protect the users real identity.

\subsection{Correlation Privacy}
Correlation Privacy is sort of a special case when it comes to the nature of the privacy threat vectors. Correlation attack is possible when using query enlargement techniques such as K-anonymity \cite{Sweeney:2002:KAM:774544.774552}. For instance, consider a user using a LBS which takes advantage of the user's mobile device's satellite positioning capabilities and requires user to send their location information to the service periodically. To protect user's identity, the LBS employs K-anonymity by sending user's data among with many other users' in the vicinity so that the user cannot be singled out of the data set. However, the anonymisation is done case-by-case. As a result of this, if the user would send the data consecutively and an adversary would be able to access these queries, the user would be included in every data set and could be easily identified with a simple interjection of the data sets. Figure ~\ref{fig:venn} illustrates, how multiple queries may possible identify the user present in all data sets.

\begin{figure}[ht!]
\centering
\floatstyle{boxed}
\includegraphics[width=90mm]{"venncorrelation".jpg}
\hbox{\scriptsize Credit:\thinspace{\small \itshape Kiia Pitkänen, used with permission}}
\caption{A user present in all data sets can be easily singled out with a simple interjection.}
\label{fig:venn}
\end{figure} 
\pagebreak
\section{Techniques for countering the privacy threats}

As there are possible privacy threats in LBS so is there techniques designed to address and mitigate their severity. \cite{Jensen2009} classify different threat countering techniques into four classes.
\begin{itemize}
\item \textbf{Query enlargement techniques} take the user's location and generalize it thus increasing the user's degree of anonymity. 
\item \textbf{Dummy-based techniques} generate fake locations and send them along with the user's actual location effectively hiding their whereabouts among the dummies.
\item \textbf{Progressive retrieval techniques} retrieve candidate query results iteratively from the server so that the user's exact location will not be revealed.
\pagebreak
\item \textbf{Transformation-based techniques} use cryptography to hide users' data, but provides users deciphering keys to decrypt the data they need. 
\end{itemize}

\subsection{Query Enlargement techniques}

Query enlargement techniques rely on the notion, that when sending location data to the server, if included is other identities from the given location, the user sending the data can not be easily identified from among other identities. The idea was first formulated by Latanya Sweeney\cite{Sweeney:2002:KAM:774544.774552} and in her paper it is applied to traditional databases. We assume that the database is made of rows of entries which consist of columns of attributes. The purpose of K-anonymity is to have a database with at least \textit{k} occurrences of a certain attribute. When someone queries the database with these attributes they will get at least \textit{k} entries which correspond to the given query, but the query can not be used to find a specific row. In this manner the data of a single user is safe but the dataset as a whole can still be used meaningfully for statistical study. To achieve a database which satisfies K-anonymity, one must first identify the so-called \textit{Quasi-identifiers} which are usually non-unique attributes in the data set but can be used together with other quasi-identifiers to form unique identifiers. To make sure that the Quasi-identifiers cannot be used to identify single entries, they either have to be \textit{suppressed}, made less distinguishable by replacing the value or part of it with a wild card notation, or \textit{generalised} so that multiple values fall to a certain broader category. Figure ~\ref{fig:kanonymity2} illustrates a small fictional anonymised database of medical records. The \textit{k} in the database is 2 which means that there are at least two of each entries with the same quasi-identifier.

\begin{figure}[H]
\centering
\floatstyle{boxed}
\includegraphics[width=120mm]{"kanonymity2".jpg}
\caption{Example of K-anonymity, where k=2 and the quasi-identifier consists of \{Race, Birth, Gender, ZIP\}. Original figure: \cite{Sweeney:2002:KAM:774544.774552} }
\label{fig:kanonymity2}
\end{figure} 

In LBS context K-anonymity for the users' identities is achieved by generalising the location in the user's query into the so-called \textit{Cloaking Range}. This is called \textit{Spatial Cloaking}. When user sends data to the server, along is sent data from other users so that the exact identity and location of the user cannot be distinguished among others \cite{Gedik2008}. To anonymise the location, an approach called \textit{Temporal Cloaking} can be used. In this approach the messages sent to the server are delayed until \textit{k} mobile users have visited the location and thus the temporal dimension of the data can not be used to identify the user as the users in the data set are not necessarily present at the time of the query.

\subsection{Dummy-based techniques}

Dummy-based techniques can be seen as the other side of the query expansion techniques. The general idea is to send the location data in a set made out of faked dummy locations, instead of multiple actual user identities and location data \cite{Kido2005}. The server then returns results for all of the locations, real and fakes as well, of which the user extracts the data associated with their real location. If this technique would be used on the case-by-case basis, the real location would be easily found as the fake location would seem to move erratically and irregularly whereas the real location would abide to real world constraints. To counter this Kido et al. provide two different ways to generate dummy locations \cite{Kido2005}. The first one called \textit{Moving in a Neighbourhood} (MN) keeps the previous generation iteration in memory and uses it to generate the next iteration so that the fake locations are moved in relation to their own previous position decreasing the possibility for an adversary to find out the real location by observing the movement patterns. The other algorithm called \textit{Moving in a Limited Neighbourhood} (MLN) requires that the user has access to other users' positional data. It works similarly to MN but if a new dummy would be created to an area which is highly congested by real users, it is generated again somewhere else instead. This is necessary because if the number of persons change in a certain area (moving etc.) there is a real possibility for the dummy location to stand out when compared to the movement of real locations \cite{Kido2005}. \par
The caveat of these proposed generation algorithms is, that they do not take the physical area in to account. This is important, as the size of the area correlates with the difficulty for the adversary to locate  the user. Increasing the number of generated dummies does not explicitly guarantee the increase in the privacy region's size. \cite{Lu2008} propose an enhanced version of the MN and LMN algorithms which distribute the dummies pseudo-randomly to a circular or a grid-based structure so that the size of the privacy region can be controlled. Grid-based distribution can also be used to reduce data transfer costs as the grid configuration can be sent in lieu of the raw data of each real and dummy location.

\subsection{Progressive Retrieval Techniques}

Progressive Retrieval Techniques are based on the idea, that the user retrieves results incrementally from the server until certain criteria, such as the sought after value is guaranteed to be in the data set or the user simply terminating the query and possibly settling for an approximation or a 'good enough' result. In this manner the user can get a query results without exposing their real location. \par

A prominent example of a progressive retrieval application is SpaceTwist \cite{SpaceTwist}. SpaceTwist is used to find \textit{k} nearest neighbours for the location \textit{q}. First the user generates a fake location \textit{q'} which is located somewhere in \textit{dist(q, q')} radius. The nearest neighbours are sought incrementally in a growing circle starting from the fake location. The algorithm maintains two numeric variables \textgamma \hspace{1mm} and \texttau. \textgamma \hspace{1mm} is used to store the distance between \textit{q} and its nearest neighbour point whereas \texttau \hspace{1mm} stores the distance between the the fake location \textit{q'} and the farthest retrieved point. The rationale for the two variables is that they are used to guarantee that the algorithm retrieves the actual \textit{k} nearest neighbours because the algorithm terminates when the condition \\ \textgamma \hspace{1mm} + \textit{dist(q, q')} \( \leq \) \texttau \hspace{1mm} is met. \par 

Figure ~\ref{fig:SpaceTwist} depicts the situation of the SpaceTwist algorithm after three iterations. The neighbours \textit{p} are in order they have been found. The algorithm would stop after finding the third \textit{p} as the distance between \textit{q'} and \textit{p\textsubscript{3}}, \texttau, is larger than the sum of \textit{dist(q, q')} and \textit{dist(q, p\textsubscript{2})}, denoted by \textgamma.

\begin{figure}[ht!]
\centering
\floatstyle{boxed}
\includegraphics[width=90mm]{"SpaceTwist".jpg}
\caption{SpaceTwist search results after three iterations}
\label{fig:SpaceTwist}
\end{figure} 

\subsection{Transformation-based Techniques}

Transformation-based Techniques maintain users' privacy by employing cryptographic functions. For example \cite{Hilbert} and \cite{Um2010} use Hilbert Curves to reduce the two-dimensional presentation of area to one-dimensional sequence of values which can then be encrypted with some encryption key and sent to server.  When the user queries for location, they transform their own location to a Hilbert value and the server returns the requested number of Hilbert values closest to the argument value. The user can then do the transformation inversely to decode the actual location data from the value. Another prominent example is Private Information Retrieval (PIR) by Ghinita et al.\cite{Ghinita2008}. PIR partitions the domain space to a grid and with every cell containing a bitmap presenting the points of interest in that cell. The bitmap is then used with random generated numbers for columns to perform modulo arithmetic which results in the quadratic residue for the corresponding cell and non-quadratic residue for others. The user can then easily find the number corresponding the the row their location resides to retrieve other near locations whereas it is computationally overwhelmingly inefficient for the adversary to do.	

\section{Evaluation of the techniques}

Obviously different approaches work for different sort of LBS. When considering social LBS such as Facebook, that offer \textit{GeoTagging} and \textit{User Tagging} possibilities most of the threat vectors are applicable to them. Unlike the techniques presented in this paper, the solutions addressing threats are not likely to be as technical as other LBS would require, but more behavioural in nature. Simply educating users about the privacy threats and how users expose themselves would do much in preserving privacy. Some exposure is not in the user's own hands necessarily, such as user tagging and, but adding options to review tagging and possibility to either accept the tag or deny it would mitigate the chances for unwanted exposure. Social LBS could also use rules for the user to define the locations  or other users they do not want to be associated with. \cite{mascetti} propose a system which leverages on GPS and other location systems. In their system the user can set up \textit{Minimum Uncertainty Regions} (MUR) in which no information will be disclosed about the user. Basically the users accepts, that the possible adversary can know that they are located within a MUR they have set up, but the adversary cannot know where exactly in the said MUR the user resides. The system allows MURs to be set for the general case or for specific users. Of course, this kind of a technical would allow other technical solutions, such as the query enlargement techniques. \par
When using K-anonymity in LBS context Gedik et al.\cite{Gedik2008} noted that the method achieved good success rates and level of anonymity while not incurring significant overheads to the performance. Though, as with all query enlargement techniques, when applying spatial or temporal cloaking the Quality of Service (QoS) may degrade as the service or the user may require more precise position or can not tolerate delays. The authors also assume, that identities are anonymous in the LBS in which K-anonymity is applied and cast doubts on technique's applicability in LBS's which require true identity. However, they argue that applications which employ pseudonyms could benefit from K-anonymity. \cite{Ghinita2008} point out that query enlargement techniques are suspectible  to correlation attacks, if the queries are done in case-by-case basis. As the technique depends on hiding users own identity and location among other users of the LBS, it may not provide as large degree of anonymity as intended when there are no other users available or the lack of users may degrade LBS's QoS. \par
Dummy-based techniques are applicable when querying for location data and the user does not want to reveal their exact location. The level of anonymity is proportional to the dummy-locations generated and the cloaking region size. The increased number of dummy locations comes with a trade-off in data transfer and performance, as the server has provide results for the real and dummy locations alike. \cite{Kido2005} reduce the data transfer costs by sending back only the data relevant to the query. Lu et al. propose an efficient way to reduce the cost with their grid-based distribution of dummy locations as then it is not required to send the raw locational data but only the grid configuration. Identity-wise Dummy-based techniques would work even with true identities as the technique masks location and movement. Though the MLN algorithm would provide better results than its simpler version, the requirement for querying other live users can introduce other security threats.\par
Progressive retrieval techniques are also used to obfuscate users' locations. While the authors of SpaceTwist\cite{SpaceTwist} admit, that in certain situations their privacy model is not as strong as in K-anonymity, its performance, applicability and deployability are its strengths and they offer different kinds of privacy guarantees. SpaceTwist offers a trade-off between the degree of anonymity and the workload as the user can define the distance between their fake location and real location. The further the fake location is defined, more neighbours are retrieved. \par 
The examples of Transformation-based Techniques offer different kind of strengths and weaknesses. While Hilbert Curve-based solutions \cite{Hilbert, Um2010} offer a very strong privacy model, they do not guarantee that when searching for the nearest neighbour the actual neighbour is found. PIR \cite{Ghinita2008} is proven to achieve perfect security and always returns accurate results. The trade-off for this is the significantly increased execution time when compared to other techniques.
 

\section{Conclusions}

Location-based services are pervasive in our contemporary society. The advent of smart phones and improvements in mobile networking has created a market for LBSs and many applications provide location-based features of their own. Even if users provide personal data for service providers more than ever before, the public interest for protecting it is also high. Ignoring privacy concerns in LBSs can have serious real life repercussions as adversaries in worst case scenarios could accurately monitor one's exact location as well as the locations and people they associate with. As the current course of development is heading towards internet connected devices ranging from common household items to cars, the privacy implications become more sever. For example an adversary could find a user's car by exploiting location data it provides and then remotely take control of it. \par
The continuing pervasiveness of LBSs is to be expected, but the threats and techniques as well as their deployment are likely to evolve. \cite{LocationPrivacy} consider usage of historical geo-referenced data as a possible future avenue for LBS, but while new opportunities present themselves so will new kinds of threats an the techniques for countering them.

% --- References ---
%
% bibtex is used to generate the bibliography. The babplain style
% will generate numeric rta eferences (e.g. [1]) appropriate for theoretical
% computer science. If you need alphanumeric references (e.g [Tur90]), use
%
% \bibliographystyle{babalpha-lf}
%
% instead.

\bibliographystyle{babplain-lf}
\bibliography{references-en}


% --- Appendices ---

% uncomment the following

% \newpage
% \appendix
% 
% \section{Esimerkkiliite}

\end{document}


