% --- Template for thesis / report with tktltiki2 class ---
% 
% last updated 2013/02/15 for tkltiki2 v1.02

\documentclass[english]{tktltiki2}

% tktltiki2 automatically loads babel, so you can simply
% give the language parameter (e.g. finnish, swedish, english, british) as
% a parameter for the class: \documentclass[finnish]{tktltiki2}.
% The information on title and abstract is generated automatically depending on
% the language, see below if you need to change any of these manually.
% 
% Class options:
% - grading                 -- Print labels for grading information on the front page.
% - disablelastpagecounter  -- Disables the automatic generation of page number information
%                              in the abstract. See also \numberofpagesinformation{} command below.
%
% The class also respects the following options of article class:
%   10pt, 11pt, 12pt, final, draft, oneside, twoside,
%   openright, openany, onecolumn, twocolumn, leqno, fleqn
%
% The default font size is 11pt. The paper size used is A4, other sizes are not supported.
%
% rubber: module pdftex

% --- General packages ---

\usepackage[utf8]{inputenc}
\usepackage[T1]{fontenc}
\usepackage{lmodern}
\usepackage{microtype}
\usepackage{amsfonts,amsmath,amssymb,amsthm,booktabs,color,enumitem,graphicx}
\usepackage[pdftex,hidelinks]{hyperref}
\usepackage{setspace}

% Automatically set the PDF metadata fields
\makeatletter
\AtBeginDocument{\hypersetup{pdftitle = {\@title}, pdfauthor = {\@author}}}
\makeatother

% --- Language-related settings ---
%
% these should be modified according to your language

% babelbib for non-english bibliography using bibtex
\usepackage[fixlanguage]{babelbib}
\selectbiblanguage{english}

% add bibliography to the table of contents
\usepackage[nottoc]{tocbibind}
% tocbibind renames the bibliography, use the following to change it back
\settocbibname{Sources}

% --- Theorem environment definitions ---

\newtheorem{lau}{Lause}
\newtheorem{lem}[lau]{Lemma}
\newtheorem{kor}[lau]{Korollaari}

\theoremstyle{definition}
\newtheorem{maar}[lau]{Määritelmä}
\newtheorem{ong}{Ongelma}
\newtheorem{alg}[lau]{Algoritmi}
\newtheorem{esim}[lau]{Esimerkki}

\theoremstyle{remark}
\newtheorem*{huom}{Huomautus}


% --- tktltiki2 options ---
%
% The following commands define the information used to generate title and
% abstract pages. The following entries should be always specified:

\title{Techniques for countering privacy threats of Location-based services}
\author{Lauri Suomalainen}
\date{\today}
\level{Seminar}
\abstract{Abstract}

% The following can be used to specify keywords and classification of the paper:

\keywords{Geo-location, Privacy, Social Networks}

% classification according to ACM Computing Classification System (http://www.acm.org/about/class/)
% This is probably mostly relevant for computer scientists
% uncomment the following; contents of \classification will be printed under the abstract with a title
% "ACM Computing Classification System (CCS):"
% \classification{}

% If the automatic page number counting is not working as desired in your case,
% uncomment the following to manually set the number of pages displayed in the abstract page:
%
% \numberofpagesinformation{16 sivua + 10 sivua liitteissä}
%
% If you are not a computer scientist, you will want to uncomment the following by hand and specify
% your department, faculty and subject by hand:
%
% \faculty{Matemaattis-luonnontieteellinen}
% \department{Tietojenkäsittelytieteen laitos}
% \subject{Tietojenkäsittelytiede}
%
% If you are not from the University of Helsinki, then you will most likely want to set these also:
%
% \university{Helsingin Yliopisto}
% \universitylong{HELSINGIN YLIOPISTO --- HELSINGFORS UNIVERSITET --- UNIVERSITY OF HELSINKI} % displayed on the top of the abstract page
% \city{Helsinki}
%


\begin{document}


% --- Front matter ---

\frontmatter      % roman page numbering for front matter

\maketitle        % title page
\makeabstract     % abstract page

\tableofcontents  % table of contents

% --- Main matter ---

\mainmatter 
\onehalfspacing
      % clear page, start arabic page numbering
\begin{abstract}


Most of the social networking services today exploit location-based data in one way or another. Services like Facebook allow users to GeoTag their current location and tag themselves and their friends. In exchange the service can use the data to offer recommendations, news and so on.  While the users of the services are often aware that they are providing personal locational data, its pervasiveness and accuracy may come as a surprise and have serious repercussions when it comes to users' real-life privacy. A malicious actor such as a burglar could exploit the data a user provides to for example find out their street address and times when they are not home.
This seminar paper reviews and evaluates several techniques used to preserve and protect users' privacy when using Location-based services. The goal for the techniques is to counter several threats Location-based Services face in such manner that users can still use said services without compromising their security.

\end{abstract}

\section{Introduction}

Location-Based Services (LBS) are applications that operate on geographical or other location based data. Ooriginal LBSs can be traced back to mid-20th century, but what is understood by LBS today is tied to the mid-21th century emergence of G3 networks and hand-held devices with GPS capability \cite{History}. Many contemporary LSBs take a form as a part of a social networking applications such as Facebook, Foursquare and Instagram. There are several ways LBSs work. They can make use of users' mobile device's GPS and provide services based on users' whereabouts when requested or continually monitor users' location \cite{LocationPrivacy}. Some services allow users to \textit{check-in} to pre-determined venues or points of interest (POI) whereas other allow more exact location with the GPS data. This practice is commonly known as \textit{GeoTagging}. \par
LBSs do not come without security risks. By publishing their location on the internet exposes users to threats in the real world. A malicious agent, for example a burglar, could monitor one user's LBS usage and e.g. determine when the user is not at their residence. Using that information the attacker could orchestrate a break-in without the risk of being caught red-handed \cite{Friedland2010}. While many applications require user to explicitly publish their locational data, users also often do it unknowingly. For example many contemporary smart phones as well as digital cameras automatically embed metadata to photographs and it can contain privacy sensitive information such as coordinates and the time the photograph was taken \cite{Friedland2010}. In some cases the user themselves does not expose their location data explicitly, but with certain LBSs supporting \textit{User Tagging}, a user's acquaintance can expose the user without them having a say for it. Thus a user can be associated with places, people and other information they would rather keep private. \par
The obvious privacy issues have been researched a lot and many techniques have been presented which address one or more possible privacy threats LBSs face \cite{LocationPrivacy}. These range from offering users options to rule how, where and when can their locational and identity data used to more technical solutions such as query enlargements and encryption-based techniques. Each of them have have their uses and shortcomings: They can only address a certain set of privacy threats and may come with pre-assumptions and computational costs. \par 
In this paper we first review threats LBSs face in section 2. Then in section 3 we take a look at the techniques designed to address different kinds of privacy threats in LSBs. Section 4 compares and evaluates the techniques in respect of their intended use as well as other techniques. Section 5 will be the concluding part summarising the state of privacy in Location-Based Services and also shortly addressing the future trends of the field. 

\section{Threats in Location-based Services}

A privacy threats in LBS can be defined as events during which an adversary can gain information about the user which they consider sensitive\cite{LocationPrivacy}. Threats fall into two major categories: Release of sensitive location information and re-identification through location information. In the first case, the identity of the user is known but the location information associated with that identity is considered sensitive and somehow made available for the adversary. In the latter case the user would like to have their identity kept secret but the adversary can exploit their location data to narrow down their possible identities in the set of other associated identities thus reducing their degree of anonymity. \cite{LocationPrivacy} list four different privacy threat vectors and \cite{Ghinita2008} add one more. The vectors are location privacy, absence privacy, co-location privacy, identity privacy and correlation privacy.

\begin{itemize}
\item \textbf{Location privacy} is the most straightforward of attack vectors. Whenever a user publishes their location they associate their identity with that location and possibly reveal information they consider sensitive. Even though a user can generalise their location it is not guaranteed to be safe. For example, a user would GeoTag themselves in a LBS whilst visiting a bar with a friend, but instead of using the exact location they would use the more general location like the part of town as the tag. Now if the aforementioned friend would happen to meet another friend at the bar too and GeoTag them both but now using the exact location of the bar, an adversary with the access of both GeoTags could easily deduct first user's exact location as well as the fact that they are accompanied by the third friend, both being pieces of information the first user did not explicitly wish to expose \cite{LocationPrivacy}.
\item \textbf{Absence Privacy} can be regarded as the other side of the location privacy. As the user publishes information about where they are at the given moment, they simultaneously expose explicitly where they are not. This threat factor also has a temporal dimension. For example, an adversary planning a break-in to a user's home can use the user's location information to estimate how long they will approximately be absent in order to avoid being caught while on the crime scene.
\item \textbf{Co-location Privacy} refers to a privacy threat in which the adversary monitors multiple users' location information in order to associate other users with each other and different locations. For example, if user A GeoTags themselves and their friend user B to a location and simultaneously user C tags themselves and their friend user D at the same place, an adversary with the access to the information of only A and C can possibly violate the privacy of both B and D as being associated with someone on the other party at the given place can be considered sensitive by users.
\item \textbf{Identity Privacy} refers to a scenario in which a user employs an obfuscated identity such as a pseudonym to protect their real identity but their identity can be revealed by associating them with a certain location. For example, the user is a 30-year-old male who is using a location-based dating service at his workplace. In the service he uses a pseudonym because he considers the fact that he is using the service to be embarrassing and thus sensitive. Now if he happened to be the only male of 30 years at his workplace and his supervisor would happen to check out the dating service and only search for users in their immediate vicinity, the supervisor would find out that the 30-year-old user is using the service during work hours effectively violating the user's privacy. In a scenario like this the pseudonym does naught to protect the users real identity.
\item \textbf{Correlation Privacy} is sort of a special case when it comes to the nature of the privacy threat vectors. Correlation attack is possible when using query enlargement techniques such as k-anonymity \cite{Sweeney:2002:KAM:774544.774552}. For instance, consider a user using a LBS which takes advantage of the user's mobile device's satellite positioning capabilities and requires user to send their location information to the service periodically. To protect user's identity, the LBS employs k-anonymity by sending user's data among with many other users' in the vicinity so that the user cannot be singled out of the data set. However, the anonymisation is done case-by-case. As a result of this, if the user would send the data consecutively and an adversary would be able to access these queries, the user would be included in every data set and could be easily identified with a simple interjection of the data sets.
\end{itemize}



\section{Techniques for countering the privacy threats}
\section{Evaluation of the techniques}




% --- References ---
%
% bibtex is used to generate the bibliography. The babplain style
% will generate numeric references (e.g. [1]) appropriate for theoretical
% computer science. If you need alphanumeric references (e.g [Tur90]), use
%
% \bibliographystyle{babalpha-lf}
%
% instead.

\bibliographystyle{babplain-lf}
\bibliography{references-en}


% --- Appendices ---

% uncomment the following

% \newpage
% \appendix
% 
% \section{Esimerkkiliite}

\end{document}


